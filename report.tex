% !TEX TS-program = pdflatex
% !TEX encoding = UTF-8 Unicode

% This is a simple template for a LaTeX document using the "article" class.
% See "book", "report", "letter" for other types of document.

\documentclass[11pt]{article} % use larger type; default would be 10pt

\usepackage[utf8]{inputenc} % set input encoding (not needed with XeLaTeX)

%%% Examples of Article customizations
% These packages are optional, depending whether you want the features they provide.
% See the LaTeX Companion or other references for full information.

%%% PAGE DIMENSIONS
\usepackage{geometry} % to change the page dimensions
\geometry{a4paper} % or letterpaper (US) or a5paper or....
% \geometry{margin=2in} % for example, change the margins to 2 inches all round
% \geometry{landscape} % set up the page for landscape
%   read geometry.pdf for detailed page layout information

\usepackage{graphicx} % support the \includegraphics command and options

% \usepackage[parfill]{parskip} % Activate to begin paragraphs with an empty line rather than an indent

%%% PACKAGE
\usepackage{mathtools}
\usepackage{listings}
\usepackage{graphicx}
\usepackage{booktabs} % for much better looking tables
\usepackage{array} % for better arrays (eg matrices) in maths
\usepackage{paralist} % very flexible & customisable lists (eg. enumerate/itemize, etc.)
\usepackage{verbatim} % adds environment for commenting out blocks of text & for better verbatim
\usepackage{subfig} % make it possible to include more than one captioned figure/table in a single float
% These packages are all incorporated in the memoir class to one degree or another...

%%% HEADERS & FOOTERS
\usepackage{fancyhdr} % This should be set AFTER setting up the page geometry
\pagestyle{fancy} % options: empty , plain , fancy
\renewcommand{\headrulewidth}{0pt} % customise the layout...
\lhead{}\chead{}\rhead{}
\lfoot{}\cfoot{\thepage}\rfoot{}

%%% SECTION TITLE APPEARANCE
\usepackage{sectsty}
\allsectionsfont{\sffamily\mdseries\upshape} % (See the fntguide.pdf for font help)
% (This matches ConTeXt defaults)

%%% ToC (table of contents) APPEARANCE
\usepackage[nottoc,notlof,notlot]{tocbibind} % Put the bibliography in the ToC
\usepackage[titles,subfigure]{tocloft} % Alter the style of the Table of Contents
\renewcommand{\cftsecfont}{\rmfamily\mdseries\upshape}
\renewcommand{\cftsecpagefont}{\rmfamily\mdseries\upshape} % No bold!
\lstset{breaklines=TRUE,language=R}
%%% END Article customizations
\title{MA691-Assignment 1\\ Statistical Simulation \& Data Analysis}
\author{Sourav Bikash\\11012338}
%\date{} % Activate to display a given date or no date (if empty),
         % otherwise the current date is printed 

\begin{document}
\maketitle
\tableofcontents
\newpage
\section{Question 1}
\subsection{Part A}
\begin{lstlisting}

y1=array();
y2=array();

x1=array();
x2=array();
x3=array();
beta1=1;
beta2=2;
beta3=3;

theta=4;
for(i in 1:50)
{
	x1[i]=runif(1);
	x2[i]=runif(1);
	x3[i]=runif(1);
	
	x1[i]=((-log(1-x1[i]))^(1/beta1))/theta;
	x2[i]=((-log(1-x2[i]))^(1/beta2))/theta;
	x3[i]=((-log(1-x3[i]))^(1/beta3))/theta;
	
	
	if(x1[i]<x3[i])
	{
		y1[i]=x1[i];
	}else{
		y1[i]=x3[i];
	}
	
	if(x2[i]<x3[i])
	{
		y2[i]=x2[i];
	}else{
		y2[i]=x3[i];
	}
}
\end{lstlisting}
\subsection{Part B}
\begin{lstlisting}

y1=array();
y2=array();

x1=array();
x2=array();
x3=array();
beta1=1;
beta2=2;
beta3=3;

theta=4;
for(i in 1:50)
{
	x1[i]=runif(1);
	x2[i]=runif(1);
	x3[i]=runif(1);
	
	x1[i]=((-log(1-x1[i]))^(1/beta1))/theta;
	x2[i]=((-log(1-x2[i]))^(1/beta2))/theta;
	x3[i]=((-log(1-x3[i]))^(1/beta3))/theta;
	
	
	if(x1[i]<x3[i])
	{
		y1[i]=x1[i];
	}else{
		y1[i]=x3[i];
	}
	
	if(x2[i]<x3[i])
	{
		y2[i]=x2[i];
	}else{
		y2[i]=x3[i];
	}
}

Y1=array();
Y2=array();
l=1;
for(i in 1:50)
{
	if(y1[i]!=y2[i])
	{
		
		Y1[l]=y1[i];
		Y2[l]=y2[i];
		l=l+1;
	}	
}
\end{lstlisting}
\section{Question 2}
\subsection{Part A}
\begin{lstlisting}

y1=array();
y2=array();

x1=array();
x2=array();
x3=array();
lambda1=1;
lambda2=2;
lambda3=3;

alpha=4;
for(i in 1:50)
{
	x1[i]=runif(1);
	x2[i]=runif(1);
	x3[i]=runif(1);
	
	x1[i]=(-(log(1-x1[i]))/lambda1)^(1/alpha);
	x2[i]=(-(log(1-x2[i]))/lambda2)^(1/alpha);
	x3[i]=(-(log(1-x3[i]))/lambda3)^(1/alpha);
	
	
	if(x1[i]<x3[i])
	{
		y1[i]=x1[i];
	}else{
		y1[i]=x3[i];
	}
	
	if(x2[i]<x3[i])
	{
		y2[i]=x2[i];
	}else{
		y2[i]=x3[i];
	}
}
\end{lstlisting}
\subsection{Part B}
\begin{lstlisting}

Y1=array();
Y2=array();
lambda1=1;
lambda2=2;
lambda3=3;

alpha=4;
i=1;
while(i<=50)
{
	x1=runif(1);
	x2=runif(1);
	x3=runif(1);
	
	x1=(-(log(1-x1))/lambda1)^(1/alpha);
	x2=(-(log(1-x2))/lambda2)^(1/alpha);
	x3=(-(log(1-x3))/lambda3)^(1/alpha);
	
	
	if(x1<x3)
	{
		y1=x1;
	}else{
		y1=x3;
	}
	
	if(x2<x3)
	{
		y2=x2;
	}else{
		y2=x3;
	}
	
	if(y1!=y2)
	{
		Y1[i]=y1;
		Y2[i]=y2;
		i=i+1;
	}
	
	
}
\end{lstlisting}
\section{Question 3}
\end{document}
